%% ========================================================================== %%
%% Title							LaTeX article header
%% ========================================================================== %%
%% File								header.tex
%%
%% Authors							Daniel Ríos Linares
%%
%% Date								March 9, 2018
%%
%% Description
%% -------------------------------------------------------------------------- %%
%%									Fancy book template with publishing purpose
%%
%% ========================================================================== %%


%%______[ USEPACKAGE STATEMENTS ]______%%
%% -------------------------------------------------------------------------- %%

% Graphics, plots and figures
\usepackage{graphicx}															% \in­clude­graph­ics CMD
\usepackage{float}																% Floating objects
\usepackage{tikz}																% TikZ support
	\usetikzlibrary{arrows,shapes,positioning}
	\usetikzlibrary{decorations.pathmorphing}
	\usetikzlibrary{decorations.markings}
	\usetikzlibrary{shapes.multipart}
		\tikzset{every text node part/.style={align=center}}
	\usetikzlibrary{intersections}
	\usetikzlibrary{quotes,angles}
	\usetikzlibrary{positioning}
	\usetikzlibrary{patterns}
	\usetikzlibrary{circuits}
	\usetikzlibrary{arrows}
	\usetikzlibrary{shapes}
	\usetikzlibrary{trees}
	\usetikzlibrary{babel}
	\usetikzlibrary{calc}
	\usetikzlibrary{fit}
	\usetikzlibrary{3d}
\usepackage[american,siunitx]{circuitikz}										% Circuit drawings
\usepackage{pgfgantt}
\usepackage{pgfplots}

% Tables
\usepackage{multirow}															% Multirow in tabular
\usepackage{tabularx}															% Defines an en­vi­ron­ment tab­u­larx with X des­ig­na­tor (para­graph col­umn)
\usepackage{booktabs}															% Table wizard
\usepackage{ltablex}																% tab­u­larx + longtable

% References
\usepackage[numbers, sort&compress]{natbib}

% Text formatting
\usepackage[hidelinks]{hyperref}													% Cross-ref­er­enc­ing com­mands (TIP: hidelinks argument for red frame)
\usepackage{glossaries}															% Glossaries
	\makeglossaries
\usepackage[bottom]{footmisc}													% Type­set­ting of foot­notes (at bottom)
\usepackage[utf8]{inputenc}														% UTF-8 encoding
\usepackage{xcolor,color}														% Fore­ground (text, rules, etc.) and back­ground colour man­age­ment
\usepackage{colortbl}
\usepackage{indentfirst}															% first paragraph indent
\usepackage[super]{nth}															% \nth{} CMD 1st, 2nd...
\usepackage{mathtools}															% Math tools
\usepackage{blindtext}															% Dummy text
\usepackage{enumitem}															% Control of list en­vi­ron­ments: enu­mer­ate, item­ize and description
\usepackage{multicol}															% Adds multicols environment that displays text until 10 columns
\usepackage{etoolbox}															% Tool­box of pro­gram­ming fa­cil­i­ties
\usepackage{setspace}															% Setstretch{n} increase multiline paragraph separation
\usepackage{fancyhdr}															% Header and footer customisation text display
\usepackage{titling}																% \theauthor, \thedate \thetitle CMDs
\usepackage{relsize}																% Math larger symbols
\usepackage{physics}																% Physics symbols
\usepackage{framed}																% Framed, shaded and leftbar environments
\usepackage{babel}																% Language formatting
\usepackage{array}
	\newcolumntype{x}[1]{>{\centering\arraybackslash\hspace{0pt}}m{#1}}

% Lists
\usepackage{listings}															% Type­set pro­grams (pro­gram­ming code) within LaTeX
\usepackage{tocloft}																% Creation of listings
\setlength{\cftfignumwidth}{3.5em}


% Fonts
\usepackage{palatino}															% Different font of LaTeX, the classic is the \usepackage{cm-super/type1ec}
\usepackage{mathrsfs}															% Vectorial fonts in math mode

% Math & physics
\usepackage{amsfonts, amsmath, amssymb, latexsym, textcomp, gensymb, mathtools}	% American Mathematical Society (AMS) fonts and math symbols
\usepackage[version=4]{mhchem}													% Chemical formulation
\usepackage{siunitx}																% Physics units
\usepackage{cancel}																% Defines \xcancel{<math mode>} that cancels terms on equations
\usepackage{esint}																% \varoiint command
\usepackage{breqn}																% Multi-line equation (auto)

% PDF tools
\usepackage{pdfpages}															% Multi-page PDF insert

% References
\usepackage[labelfont = bf, justification=centering, margin=2cm]{caption}			% Customisation of the cap­tions in float­ing en­vi­ron­ments
\usepackage[labelfont = md, justification=centering, margin=0.5cm]{subcaption}		% Subfigure support
\usepackage{titlesec,titletoc}													% Alternative section titles
	\setcounter{tocdepth}{2} % TOC show until 1: section, 2: subsection, 3: subsubsection...
\usepackage[nottoc,numbib]{tocbibind}											% Bibliography no TOC


%%______[ MATH ]______%%

% Imaginary unit
\DeclareMathAlphabet{\mymathbb}{U}{BOONDOX-ds}{m}{n}
\newcommand{\imag}{\ensuremath{\mymathbb{j}}}


%%______[ SPACING ]______%%
%% -------------------------------------------------------------------------- %%

% Rows
\setlength{\extrarowheight}{0.1cm}
% General
\renewcommand{\arraystretch}{1.05}
% Remove rubber spacing
\raggedbottom


%%______[ INTERNAL DEFINITIONS ]______%%
%% -------------------------------------------------------------------------- %%

\makeatletter
\let\runningauthor\@author
\let\runningtitle\@title
\let\runningdate\@date
\makeatother


%%______[ HEADER & FOOTER CUSTOMISATION ]______%%
%% -------------------------------------------------------------------------- %%

% Default
\fancyfoot[RO]{\thepage} % Footer right odd: page numbering
\fancyfoot[LE]{\thepage} % Footer left even: page numbering
\fancyfoot[LO]{\runningauthor} % Footer left even: Author name
\fancyfoot[RE]{\runningtitle} % Footer right even: Title of the document
\fancyfoot[CO]{} % Footer center odd: empty
\fancyfoot[CE]{} % Footer center even: empty
\fancyhead[RO]{\nouppercase\leftmark} % Header right odd: Chapter
\fancyhead[LE]{\nouppercase\rightmark} % Header left even: Sectioning
\fancyhead[RE]{} % Header left even: empty
\fancyhead[LO]{} % Header left even: empty
\fancyhead[CO]{} % Header center odd: empty
\fancyhead[CE]{} % Header center even: empty

% Chapter page
\fancypagestyle{plain}{%
	\pagestyle{fancy}
}

% Header and footer rule widths
\renewcommand{\headrulewidth}{0.5pt}											% Header rule width
\renewcommand{\footrulewidth}{0.5pt}											% Footer rule width

% Apply header and footer customisation
\pagestyle{fancy}																% Apply header and footer customisation


%%______[ TOC CUSTOMISATION (labeling format ]______%%
%% -------------------------------------------------------------------------- %%

% Part
\titlecontents{part}%
  [0pt]% left aligned
  {\vspace{1cm} \hfill\bfseries\scshape\partname~}% above code
  {\large\bfseries\scshape\partname~\thecontentslabel.~}% Number entry style
  {\large\bfseries\scshape}% Numberless entry style
  {\hfill\bfseries\contentspage}% Horizontal filler format

% Chapter
\titlecontents{chapter}%
  [0pt]% left aligned
  {}% above code
  {\large\bfseries\scshape\chaptername~\thecontentslabel.}% Number entry style
  {\large\bfseries\scshape}% Numberless entry style
  {\hfill\bfseries\contentspage}% Horizontal filler format

% Section
\titlecontents{section}%
  [1em]% left aligned
  {}% above code
  {\thecontentslabel{}.~}% Number entry style
  {}% Numberless entry style
  {\hfill\contentspage}% Horizontal filler format

% Subsection
\titlecontents{subsection}%
  [3em]% left aligned
  {\itshape}% above code
  {\itshape\thecontentslabel.~}% Number entry style
  {}% Numberless entry style
  {\dotfill\normalfont\contentspage }% Horizontal filler format

% Subsubsection
\titlecontents{subsubsection}%
  [4em]% left aligned
  {}% above code
  {\thecontentslabel.~}% Number entry style
  {}% Numberless entry style
  {\dotfill\contentspage }% Horizontal filler format


%%______[ TOC CUSTOMISATION (contents format) ]______%%
%% -------------------------------------------------------------------------- %%

% Restart references
\usepackage{remreset}
% Remove section restart after chapter
%	LEGACY \makeatletter\@removefromreset{section}{chapter}\makeatother

% Self-made sectioning names (Babel overrides)
\def\acknowledgementsname{Acknowledments}
\def\summaryname{Summary}
\def\summarykeywordsname{Keywords}
\def\summaryabstractname{Abstract}
\def\sourcecodename{Code}
\def\figurename{Figure}
\def\tablename{Table}

% List of... customisation
\def\listofsourcecodename{List of \sourcecodename}
\def\listofreferencesname{}

% Self-made floats
\usepackage{newfloat}
% Source code
\DeclareFloatingEnvironment[fileext=frm,listname={\listofsourcecodename},placement={!ht},name=\sourcecodename]{sourcecode}
\captionsetup[sourcecode]{labelfont=bf}

% Chapter with rule below in table of contents
\renewcommand{\chapter}[1]{
	\cleardoublepage
	\stepcounter{chapter}
	\addcontentsline{toc}{chapter}{\chaptername~\thechapter:~#1}
	\addtocontents{toc}{\vskip-6pt\par\noindent\hrulefill\par}
	\def\leftmark{\chaptername~\thechapter:~#1}
	\roundedlabel{#1}{\chaptername}
	\indent
}

% Appendix appearing in the table of contents
\let\oldappendix\appendix
\renewcommand{\appendix}{
	\cleardoublepage
	~\\*
	\vspace{3cm}
	~\\*
	\noindent
	\begin{tikzpicture}[scale=\textwidth/2cm]
	\node [inner sep = 0] (O) at (0,0) {};
	\node [text width = 5in, execute at begin node={\setlength{\baselineskip}{3em}}] (F) at (0,-0.25) {\huge{\textsc{\appendixname}}};
	\draw
		(-0.75,0)--(+0.75,0)
		arc (270:90:-0.25)
		(-0.75,0)
		arc (90:270:0.25)
		--(0.75,-0.5)
		;
	\end{tikzpicture}
	~\\*
	\vspace{1cm}
	~\\*
	\renewcommand{\leftmark}{\appendixname}
	\renewcommand{\thesection}{\Alph{section}}
	\oldappendix
	\addtocontents{toc}{\vskip-6pt\par\noindent\protect\hfill\par}
	\addtocontents{toc}{\vskip-6pt\par\noindent\protect\hfill\par}
	\addcontentsline{toc}{chapter}{\appendixname}
	\addtocontents{toc}{\vskip-6pt\par\noindent\protect\hrulefill\par}
}

% Space between entries in table of contents
\makeatletter
\pretocmd{\chapter}{\addtocontents{toc}{\protect\addvspace{10\p@}}}{}{}
\pretocmd{\section}{\addtocontents{toc}{\protect\addvspace{4\p@}}}{}{}
\pretocmd{\subsection}{\addtocontents{toc}{\protect\addvspace{4\p@}}}{}{}
\pretocmd{\subsubsection}{\addtocontents{toc}{\protect\addvspace{4\p@}}}{}{}
\makeatother

% Babel language integration renaming (english)
\addto\captionsenglish{%
	\renewcommand{\partname}{Parte}
	\renewcommand{\chaptername}{Capítulo}
	\renewcommand{\contentsname}{Índice}
	\renewcommand{\acknowledgementsname}{Agradecimientos}
	\renewcommand{\summaryname}{Resumen}
	\renewcommand{\summarykeywordsname}{Palabras clave}
	\renewcommand{\summaryabstractname}{Resumen}
	\renewcommand{\sourcecodename}{Código fuente}
	\renewcommand{\figurename}{Figura}
	\renewcommand{\tablename}{Tabla}
	\renewcommand{\listofsourcecodename}{Lista de código fuente}
	\renewcommand{\listofreferencesname}{Bibliografía}
	\renewcommand{\listfigurename}{Lista de figuras}
	\renewcommand{\listtablename}{Lista de tablas}
	\renewcommand{\refname}{Bibliografía}
	\renewcommand{\appendixname}{Apéndice}
}

% Sectioning labels format
\renewcommand{\thechapter}{\Roman{chapter}}
\renewcommand{\thepage}{\arabic{page}}  
\renewcommand{\thesection}{\arabic{section}}
\renewcommand{\thesubsection}{\arabic{section}.\arabic{subsection}}
\renewcommand{\thesubsubsection}{\arabic{section}.\arabic{subsection}.\arabic{subsubsection}}
\renewcommand{\thetable}{\arabic{section}.\arabic{table}}   
\renewcommand{\thefigure}{\arabic{section}.\arabic{figure}}
\renewcommand{\thesourcecode}{\arabic{section}.\arabic{sourcecode}}

% List of... (modified)
% Source code
\let\oldlistoftables\listoftables
\renewcommand{\listoftables}{
	\addtocontents{toc}{\vskip-6pt\par\noindent\protect\hfill\par}
	\renewcommand{\chaptermark}{\listtablename}
	\renewcommand{\leftmark}{\listtablename}
	\addcontentsline{toc}{chapter}{\listtablename}
	\oldlistoftables
}
\let\oldlistoffigures\listoffigures
\renewcommand{\listoffigures}{
	\addtocontents{toc}{\vskip-6pt\par\noindent\protect\hfill\par}
	\renewcommand{\rightmark}{\listfigurename}
	\renewcommand{\leftmark}{\listfigurename}
	\addcontentsline{toc}{chapter}{\listfigurename}
	\oldlistoffigures
}
\let\oldlistofsourcecodes\listofsourcecodes
\renewcommand{\listofsourcecodes}{
	\renewcommand{\rightmark}{\listofsourcecodename}
	\oldlistofsourcecodes
}
\newcommand{\listofreferences}{
	\renewcommand{\chapter}[2]{\addcontentsline{toc}{chapter}{\listofreferencesname}}
	\renewcommand{\rightmark}{\refname}
	~\\
	~\\
	~\\
	~\\
	~\\
	\noindent\textbf{\Huge\listofreferencesname}
	~\\
	~\\
}

%%______[ INTERESTING COMMANDS ]______%%
%% -------------------------------------------------------------------------- %%

% Rounded chapter label
\newcommand{\roundedlabel}[2]{
	\ifx\hfuzz#2\hfuzz
	~\\*
	\vspace{3cm}
	~\\*
	\noindent
	\begin{tikzpicture}[scale=\textwidth/2cm]
	%\node [inner sep = 1cm] (O) at (0,0) {\Large{\textsc{#2 \thechapter}}};
	\node [text width = 5in, execute at begin node={\setlength{\baselineskip}{3em}}] (F) at (0,-0.25) {\huge{\textsc{#1}}};
	\draw
		(-0.75,0)--(+0.75,0)
		arc (270:90:-0.25)
		(-0.75,0)
		arc (90:270:0.25)
		--(0.75,-0.5)
		;
	\end{tikzpicture}
	~\\*
	\vspace{1cm}
	~\\*
	\else
	~\\*
	\vspace{3cm}
	~\\*
	\noindent
	\begin{tikzpicture}[scale=\textwidth/2cm]
	\node [inner sep = 1cm] (O) at (0,0) {\Large{\textsc{#2 \thechapter}}};
	\node [text width = 5in, execute at begin node={\setlength{\baselineskip}{3em}}] (F) at (0,-0.25) {\huge{\textsc{#1}}};
	\draw
		(O)--(+0.75,0)
		arc (270:90:-0.25)
		(O)--(-0.75,0)
		arc (90:270:0.25)
		--(0.75,-0.5)
		;
	\end{tikzpicture}
	~\\*
	\vspace{1cm}
	~\\*
	\fi
}

% Create a "chapter" with empty style with acknowledgements
\newcommand{\acknowledgements}[1]{
	\addtocontents{toc}{\vskip-6pt\par\noindent\protect\hfill\par}
	\addcontentsline{toc}{chapter}{\acknowledgementsname}
	\roundedlabel{\acknowledgementsname}{}
	\thispagestyle{empty}
	#1
	\cleardoublepage
}

% Create a "chapter" with empty style with keywords and summary of the contents
\newcommand{\summary}[2]{
	\addtocontents{toc}{\vskip-6pt\par\noindent\protect\hfill\par}
	\addcontentsline{toc}{chapter}{\summaryname}
	\roundedlabel{\summaryname}{}
	\thispagestyle{empty}
	\textbf{\summarykeywordsname:} #1 \\ \\
	\textbf{\summaryabstractname} \\
	~\\
	#2
	\cleardoublepage
}

% Main full cover page with a title <title1>, a subtitle <title2>, a subsubtitle <title3> and a <date> display
\newcommand{\coverpage}[9] {
\begin{titlepage}
\pagenumbering{roman}
\setcounter{page}{1}
\begin{center}
% Background image
\ifx\hfuzz#8\hfuzz
\else\tikz[remember picture,overlay] \node[opacity=#9,inner sep=0pt] at (current page.center){\includegraphics[width=\paperwidth,height=\paperheight]{#8}};
\fi
~\\
[10mm]
{\huge{#1}} \\																	% Department
[7mm]
{\Large{#2}} \\																	% Subdepartment
[7mm]
% Background image
\ifx\hfuzz#3\hfuzz\else
    \begin{figure}[H]
	\centering\includegraphics[width=2in]{#3}
	\end{figure}
\fi
~\\
[2mm]
{\centering\Huge{\textsc{#4}}} \\												% Title
[14mm]
{\Large{#5}} \\																	% Title 2
[10mm]
{\large #6} \\																	% Date
\end{center}
\vfill
\begin{flushright}
#7																				% Authors
\end{flushright}
\line(1,0){480} \\
\end{titlepage}
}


%%______[ CAPTION FORMAT ]______%%
%% -------------------------------------------------------------------------- %%

\renewcommand{\theequation}{\Roman{chapter}-\arabic{section}.\arabic{equation}}	% Equation
\renewcommand{\thefigure}{\Roman{chapter}-\arabic{section}.\arabic{figure}}		% Figure
\renewcommand{\thetable}{\Roman{chapter}-\arabic{section}.\arabic{table}}						% Table
\renewcommand{\thesourcecode}{\Roman{chapter}-\arabic{section}.\arabic{sourcecode}}



%%______[ COLOURS ]______%%
%% -------------------------------------------------------------------------- %%

\definecolor{circuitcolor}{rgb}{0.05, 0.05, 0.05}
\definecolor{shadecolor}{rgb}{1,0.8,0.3}
\definecolor{mauve}{rgb}{0.58,0,0.82}
\definecolor{textcolor}{rgb}{0, 0, 0}
\definecolor{gray}{rgb}{0.5,0.5,0.5}
\definecolor{ltgray}{rgb}{0.2,0.2,0.2}
\definecolor{dkgreen}{rgb}{0,0.6,0}
\definecolor{awesome}{rgb}{1.0,0.13,0.32}


%%______[ LISTING SETTINGS ]______%%
%% -------------------------------------------------------------------------- %%

\lstloadlanguages{Perl}
\lstset{
	aboveskip=3mm,belowskip=3mm,													% Vertical space before and after the code
	basicstyle={\scriptsize\ttfamily},											% Size & family properties of the font
	breaklines=true,breakatwhitespace=true,										% Break large lines at white space
	columns=flexible,															% Large lines splitted
	frame=single,																% Frame around the code (tb : top & bottom, single, top, bottom...)
	numbers=left,numberstyle=\tiny\color{gray},									% Numbers at the left (with format)
	showstringspaces=false,														% Don't overwrite space with other symbol
	tabsize=4,																	% Common 4 char tab size
	xleftmargin=0.5in,xrightmargin=0.42in,										% Margin fix
	% Colour settings (typical)
	keywordstyle = \color{blue},
	commentstyle = \color{dkgreen},
	stringstyle = \color{mauve},
	postbreak = \raisebox{0ex}[0ex][0ex]{\ensuremath{\color{dkgreen}\hookrightarrow\space}}, % Huge line to multiple single line
}

% Language: LaTeX (extended)
\lstdefinelanguage{LaTeX}{
	language=TeX, morekeywords={begin},
	% TikZ internal
	classoffset=1, % Start new class
	morekeywords={node, draw, path, node},
	keywordstyle=\color{awesome},
	classoffset=0,
	classoffset=1, % Start new class
	morekeywords={label,caption},
	keywordstyle=\color{orange},
	classoffset=0,
}


%%______[ TikZ GENERAL SETTINGS ]______%%
%% -------------------------------------------------------------------------- %%

\tikzset{font = {\fontsize{10pt}{12}\selectfont}}


%%______[ circuitikz custom symbols ]______%%
%% -------------------------------------------------------------------------- %%


%%%%%%%%%%%%%%%%%%%%%%%%% PRESET %%%%%%%%%%%%%%%%%%%%%%%%%%%
\tikzset{
	ctikzST/.style = {
		% American format
		american voltages,
		american resistors,
		% General length
		/tikz/circuitikz/bipoles/length=1.2cm,
		% TikZ settings
		line width = 0.75,
		/tikz/circuitikz/bipoles/thickness=1,
		inner sep = 0pt,
		color = red,
	}
}

%%______[ PGFPLOT-AXIS PLOT SETTINGS ]______%%
%% -------------------------------------------------------------------------- %%

% Physics (without box, legend right, 0.4 x 0.3, limits enlarged)
\pgfplotsset{
    physics/.style={
        %axis x line = middle,
        %axis y line = middle,
        %enlarge x limits=0.05,
        %enlarge y limits=0.1,
        font = \footnotesize,
        %yticklabel style={/pgf/number format/.cd,fixed,precision=3},
        %every axis x label/.style={at={(current axis.right of origin)},anchor=west},
        %every axis y label/.style={at={(current axis.above origin)},anchor=south},
        every axis x label/.style={at={(axis description cs:0.5,-0.1)},anchor=north},
        %every axis y label/.style={at={(axis description cs:-0.2,.5)},rotate=90,anchor=south},
        ylabel near ticks,
        scaled y ticks = false,
        legend style={at={(1,1)},anchor=north east},
        cycle list name = auto,
        no marks,
        width = 0.4*\textwidth, height = 0.3*\textwidth,
        /pgf/number format/use comma,
        /pgf/number format/1000 sep={},
		every plot/.append style = {line width = 3pt},
    }
}




